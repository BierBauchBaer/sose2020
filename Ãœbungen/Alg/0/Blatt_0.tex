\documentclass[11pt,a4paper]{scrartcl}

\newcommand{\blatt}{0}
\newcommand{\fach}{Algebra II}
\usepackage[utf8]{inputenc}
% Sonderzeichen
\usepackage{amssymb}
\usepackage{amsmath}
\usepackage{amsfonts}
\usepackage{dsfont}
% Formeln, ...
\usepackage[german]{babel}
% Deutsch
\usepackage{cmbright}
% Font
\usepackage{graphicx}
% Includegraphics
\usepackage{fancyhdr}
% Pagestyle
\usepackage{titlesec}
% Titleformat
\usepackage[left=2cm,right=2cm,top=2.5cm,bottom=3cm]{geometry}
% Rand
\usepackage{xcolor}
\usepackage{color}
% Farbe in Tabellen und Plots
\usepackage{float}
% figures mit [h]
\usepackage{capt-of}
% Captions mit captionof{<figure/table>}{<Name>}
\usepackage{pgfplotstable}
% Tabellen aus .dat
\usepackage{mathtools}
\mathtoolsset{showonlyrefs}
\everymath{\displaystyle}

\titleformat{\section}{\normalfont}{\large\bfseries\blatt.\thesection}{8pt}{\normalfont}

\renewcommand{\thesubsection}{\alph{subsection}}
\titleformat{\subsection}{\normalfont}{(\thesubsection)}{8pt}{\normalfont}
% Sections betiteln

\pagestyle{fancy}
\rhead{\begin{tabular}{l r}
  Falk Trittschanke & 10011108\\
  Lukas Hantzko     & 10011048
\end{tabular}}
\lhead{\begin{tabular}{l}
  Hausübung \blatt\\
  {\bfseries \fach}
\end{tabular}}
% Header

\setlength\parindent{0pt}

% Symbole
\newcommand{\mE}[1]{\text{e}^{#1} }
\newcommand{\mI}{\text{i} }
\newcommand{\mQ}{\mathbb{Q} }
\newcommand{\mR}{\mathbb{R} }
\newcommand{\mC}{\mathscr{C}}
\newcommand{\mZ}{\mathbb{Z}}
\newcommand{\cC}{\mathbb{C}}
% Operatoren
\DeclareMathOperator{\tr}{Tr}
\newcommand*\Laplace{\mathop{}\!\mathbin\bigtriangleup}
\newcommand*\Nabla{\mathop{}\!\mathbin\bigtriangledown}
\newcommand*\DAlambert{\mathop{}\!\mathbin\Box}
% Klammern
\newcommand{\nOf}[1]{\left( #1 \right)}
\newcommand{\cOf}[1]{\left\{ #1 \right\}}
\newcommand{\eOf}[1]{\left[ #1 \right]}
\newcommand{\vOf}[1]{\left\lvert #1 \right\rvert}
\newcommand{\aOf}[1]{\left\langle #1 \right\rangle}
% Ableitungen
\newcommand{\dv}[2]{\frac{\text{d} #1}{\text{d} #2}}
% Besonderes
\newcommand{\scP}[2]{\aOf{#1 , #2}}


\begin{document}
\section{Was besagt der Hauptsatz der Galoistheorie? Geben Sie ein Anwendungsbeispiel.}
Der Hauptsatz der Galoistheorie besagt, dass die Untergruppen der Galoisgruppe einer Körpererweiterung den Zwischenkörpern der Körpererweiterung entsprechen. Ein Anwendungsbeispiel ist, Zwischenkörper zu finden.
\section{Bestimmen sie $\gal\nOf{f\nOf{x},\mQ}$ in den folgenden Fällen:}
\subsection{$f\nOf{x} = x^3-2 $}
Drei Nullstellen: $\alpha_1 = \sqrt[3]{2}$, $\alpha_2 = \sqrt[3]{2} \mE{\mI  \frac{2\pi}{3}}$, $\alpha_3 = \sqrt[3]{2} \mE{\mI 2 \frac{2\pi}{3}}$\\
Zerfällungskörper: $\mQ\nOf{\alpha_1,\alpha_2,\alpha_3}$\\
Permutation der Nullstellen (3)\\
$\rightarrow$ $\gal\nOf{f\nOf{x},\mQ} = S_3$
\subsection{$f\nOf{x} = x^4-2$}
Vier Nullstellen: $\alpha_1 = \sqrt[4]{2}$, $\alpha_2 = \sqrt[4]{2} \mE{\mI \frac{2\pi}{4}}$, $\alpha_3 = \sqrt[4]{2} \mE{\mI 2 \frac{2\pi}{4}}$, $\alpha_4 = \sqrt[4]{2} \mE{\mI 3 \frac{2\pi}{4}}$\\
Zerfällungskörper: $\mQ\nOf{\mI,\sqrt[4]{2}}$ \\
Permutation der Nullstellen (8) \\
$\rightarrow$ $\gal\nOf{f\nOf{x},\mQ} = D_4$
\subsection{$f\nOf{x} = x^4 - 4 x^2 + 2$}
Vier Nullstellen: $y=x^2$ $y_{\pm} = 2 \pm \sqrt{4-2}$ $x_{1,2,3,4} = \pm \sqrt{2 \pm \sqrt{2}}$ \\
Zerfällungskörper: $\mQ\nOf{\sqrt{2 + \sqrt{2}}}$ \\
Die Galoisgruppe ist $S_2 \times S_2$
\subsection{$f\nOf{x} = x^3 - 3x + 1$}
Nullstellen: \\
Zerfällungskörper: \\
Galoisgruppe:
\section{Für Beispiele in 2: Explizit}
\subsection{Verband der Untergruppen $\gal\nOf{f\nOf{x},\mQ}$}
\subsection{Verband der Zwischenkörper $\mQ \subset E$}
\end{document}
