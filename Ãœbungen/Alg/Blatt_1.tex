\documentclass[11pt,a4paper]{scrartcl}

% Allgemein
\newcommand{\blatt}{1}
\newcommand{\fach}{Algebra II}
\usepackage[utf8]{inputenc}
% Sonderzeichen
\usepackage{amssymb}
\usepackage{amsmath}
\usepackage{amsfonts}
\usepackage{dsfont}
% Formeln, ...
\usepackage[german]{babel}
% Deutsch
\usepackage{cmbright}
% Font
\usepackage{graphicx}
% Includegraphics
\usepackage{fancyhdr}
% Pagestyle
\usepackage{titlesec}
% Titleformat
\usepackage[left=2cm,right=2cm,top=2.5cm,bottom=3cm]{geometry}
% Rand
\usepackage{xcolor}
\usepackage{color}
% Farbe in Tabellen und Plots
\usepackage{float}
% figures mit [h]
\usepackage{capt-of}
% Captions mit captionof{<figure/table>}{<Name>}
\usepackage{pgfplotstable}
% Tabellen aus .dat
\usepackage{mathtools}
\mathtoolsset{showonlyrefs}
\everymath{\displaystyle}

\titleformat{\section}{\normalfont}{\large\bfseries\blatt.\thesection}{8pt}{\normalfont}

\renewcommand{\thesubsection}{\alph{subsection}}
\titleformat{\subsection}{\normalfont}{(\thesubsection)}{8pt}{\normalfont}
% Sections betiteln

\pagestyle{fancy}
\rhead{\begin{tabular}{l r}
  Falk Trittschanke & 10011108\\
  Lukas Hantzko     & 10011048
\end{tabular}}
\lhead{\begin{tabular}{l}
  Hausübung \blatt\\
  {\bfseries \fach}
\end{tabular}}
% Header

\setlength\parindent{0pt}

% Symbole
\newcommand{\mE}[1]{\text{e}^{#1} }
\newcommand{\mI}{\text{i} }
\newcommand{\mQ}{\mathbb{Q} }
\newcommand{\mR}{\mathbb{R} }
\newcommand{\mC}{\mathscr{C}}
\newcommand{\mZ}{\mathbb{Z}}
\newcommand{\cC}{\mathbb{C}}
% Operatoren
\DeclareMathOperator{\tr}{Tr}
\newcommand*\Laplace{\mathop{}\!\mathbin\bigtriangleup}
\newcommand*\Nabla{\mathop{}\!\mathbin\bigtriangledown}
\newcommand*\DAlambert{\mathop{}\!\mathbin\Box}
% Klammern
\newcommand{\nOf}[1]{\left( #1 \right)}
\newcommand{\cOf}[1]{\left\{ #1 \right\}}
\newcommand{\eOf}[1]{\left[ #1 \right]}
\newcommand{\vOf}[1]{\left\lvert #1 \right\rvert}
\newcommand{\aOf}[1]{\left\langle #1 \right\rangle}
% Ableitungen
\newcommand{\dv}[2]{\frac{\text{d} #1}{\text{d} #2}}
% Besonderes
\newcommand{\scP}[2]{\aOf{#1 , #2}}


% Matrizen etc.
\newcommand{\tauMat}{\left(\begin{array}{r r} 1 & 0 \\ 0 & -1\end{array}\right)}
\newcommand{\idMat}{\left(\begin{array}{r r} 1 & 0 \\ 0 & 1\end{array}\right)}
\newcommand{\sigMat}{\left(\begin{array}{r r} \cos \nOf{\phi} & \sin \nOf{\phi} \\ -\sin \nOf{\phi} & \cos \nOf{\phi} \end{array}\right)}
\newcommand{\sigiMt}{\left(\begin{array}{r r} \cos \nOf{i\cdot\phi} & \sin \nOf{i\cdot\phi} \\ -\sin \nOf{i\cdot\phi} & \cos \nOf{i\cdot\phi} \end{array}\right)}

% Text
\begin{document}
\section{$n\in \mN$, $n > 3$. $\sigma$ Drehung $\mR^2$ um $\phi = \frac{2\pi}{n}$. $\tau$ Spiegelung an $y$-Achse. $D_n = \aOf{\sigma,\tau}$}
Im folgenden wird $x\in \mR^2$ angenommen.
\subsection{Es gilt $\ord\nOf{\sigma} = n$, $\ord\nOf{\tau} = 2$ und $\sigma\tau\sigma = \tau$.}
$\ord\nOf{\sigma} = n$, da:
\begin{equation}
	\sigma^i\nOf{x} = \sigiMt \cdot x
\end{equation}
Und damit:
\begin{equation}
	\sigma^n\nOf{x} = \idMat \cdot x = x \quad \sigma^n = \id
\end{equation}
Für $\tau$:
\begin{equation}
	\tau\nOf{x} = \tauMat\cdot x \quad \tau^2\nOf{x} = \left(\begin{array}{r r} 1 & 0 \\ 0 & 1\end{array}\right)\cdot x = x \quad \tau^2 = \id
\end{equation}
Damit ist $\ord\nOf{\tau} = 2$.
\begin{equation}
	\sigma\tau\sigma\nOf{x} = \sigMat \tauMat \sigMat \cdot x = \tauMat \cdot x = \tau\nOf{x}
\end{equation}
Es gilt also auch $\sigma\tau\sigma = \tau$

\subsection{Es gilt $D_n = \sDf{\tau^i\sigma^j}{i\in\cOf{0,1}, j\in\cOf{0,1,\dots,n-1}}$.}
Da $D_n$ nach dem ersten Aufgabenteil abelsch ist, sind alle Elemente von der Form $D_n = \sDf{\tau^i\sigma^j}{i\in\mN, j\in\mN}$. Da die beiden erzeugenden Elemente eine Ordnung von $n$ und $2$ haben, ist 

\subsection{$D_n$ hat $n$ Elemente und ist nicht kommutativ.}
$D_n$ ohne Spiegelung ist reguläres $n$-Eck. Spiegelungen an der y-Achse sind irrelevant.

\section{$K$ Körper und $f\nOf{x}\in K \eOf{x}$ irreduzibel. $a$ Nullstelle von $f\nOf{x}$ in Erweiterungskörper von $K$.}

\subsection{Beweisen Sie: Ist auch $f\nOf{a + 1} = 0$, so gilt $\car\nOf{K} > 0$.}
Es gilt:
\\ Gelte nun weiter $\car\nOf{K} = p$ und $a^p - a \in K$.
\subsection{Beweisen Sie, dass $f\nOf{x} = x^p - x - \nOf{a^p - a}$ gilt.}

\subsection{Beweisen Sie, dass die Erweiterung $K\nOf{a}/K$ galoissch ist.}

\subsection{Beweisen Sie, dass $\aut\nOf{K\nOf{a};K}$ zyklisch von Ordnung $p$ ist.}

\section{$\mC\nOf{x}$ rationale Funktionen über $\mC$. In $\aut\nOf{\mC\nOf{x};\mC}$, betrachte $\sigma$, $\tau$ mit $\sigma\nOf{x} = - x$ und $\tau\nOf{x} = \mI x^{-1}$. $G = \aOf{\sigma,\tau}\subset \aut\nOf{\mC\nOf{x};\mC}$.}

\subsection{Beweisen Sie, dass $G$ endlich ist. Welche Ihnen bekannte Gruppe ist $G$?}

\subsection{Beweisen Sie, dass $\fix\nOf{\mC\nOf{x};G}$ rationaler Fkt-Körper über $\mC$; $\fix\nOf{\mC\nOf{x};G} = \mC\nOf{y}$ mit $y\in\mC\nOf{x}$. $y$ angeben.}

\end{document}
