\section{Sei $K$ ein Körper, und sei $m\in K$. Zeigen Sie:}
\subsection{Die Matrizen der Form $A$ bilden einen kommutativen Unterring $L_m$ von $M\nOf{2 \times 2,K}$.}
\begin{equation}
	A = \cOf{\nOf{\begin{array}{c c}a&b\\ mb&a\end{array}}}
\end{equation}
Als Teilmenge gilt Assoziativität, Distributivität und Kommutativität für $+$ automatisch. Bleibt zu zeigen, dass $L_m$ abgeschlossen ist unter $+$ und $\cdot$, und die Kommutativität von $\cdot$ in $L_m$:
\begin{align}
	\nOf{\begin{array}{c c}a&b\\ mb&a\end{array}} \cdot \nOf{\begin{array}{c c}c&d\\ md&c\end{array}} &= \nOf{\begin{array}{c c}ac+bdm&bc+ad\\ m\nOf{bc+ad}&ac+bdm\end{array}} \in A \\
	\nOf{\begin{array}{c c}a&b\\ mb&a\end{array}} + \nOf{\begin{array}{c c}c&d\\ md&c\end{array}} &= \nOf{\begin{array}{c c}a+c&b+d\\ m\nOf{b+d}&a+c\end{array}} \in A \\
	\nOf{\begin{array}{c c}c&d\\ md&c\end{array}} \cdot \nOf{\begin{array}{c c}a&b\\ mb&a\end{array}}&= \nOf{\begin{array}{c c}ac+bdm&bc+ad\\ m\nOf{bc+ad}&ac+bdm\end{array}} = \nOf{\begin{array}{c c}a&b\\ mb&a\end{array}} \cdot \nOf{\begin{array}{c c}c&d\\ md&c\end{array}}
\end{align}

\subsection{$L_m$ ist genau dann ein Körper, wenn $m$ kein Quadrat in $K$ ist.}
Sei $m=c^2$ ein Quadrat in $K$, wähle $a$ beliebig und $b = \frac{a}{c}\in K$, da $K$ Körper ist. Dann ist $L_m$ kein Körper, da:
\begin{equation}
	\det \nOf{\begin{array}{c c}a&b\\ mb&a\end{array}} = a^2 - m b^2 = a^2 - a^2 = 0
\end{equation}
also $\nOf{\begin{array}{c c}a&b\\ mb&a\end{array}}$ nicht invertierbar.\\
Ist $L_m$ kein Körper, so existieren $a,b\in K$, sodass $\nOf{\begin{array}{c c}a&b\\ mb&a\end{array}}$ nicht invertierbar ist, also folglich:
\begin{equation}
	\det \nOf{\begin{array}{c c}a&b\\ mb&a\end{array}} = a^2 - m b^2 \stackrel{!}{=} 0
\end{equation}
Damit folgt, dass $m$ ein Quadrat ist.
\begin{equation}
	a^2 = mb^2 \quad \frac{a^2}{b^2} = \frac{a}{b}^2 = m
\end{equation}

\subsection{Ist $L_m$ ein Körper und $K=\mF_p$ mit einer ungeraden Primzahl $p$, so gilt $L_m \isom \mF_{p^2}$.} Nach Algebra I Satz 5.60 sind je zwei Körper mit $p^n$ Elementen isomorph. Da wir eine bijektive Abbildung von $L_m$ nach $\mF_p\times\mF_p$ finden, ist $\vOf{L_m} = p^2$. Somit wäre die Isomorphie bewiesen. Die bijektive Abbildung ist:
\begin{equation}
	f: L_m \to \mF_p^2; \nOf{\begin{array}{c c}a&b\\ mb&a\end{array}} \mapsto (a,b)
\end{equation}
Injektivität:
\begin{equation}
	\nOf{a,b} \in \mF_p^2 \Rightarrow f\nOf{\begin{array}{c c}a&b\\ mb&a\end{array}} = \nOf{a,b}
\end{equation}
Surjektivität:
\begin{equation}
	\nOf{a,b} = \nOf{c,d} \Rightarrow a=c,b=d \Rightarrow \nOf{\begin{array}{c c}a&b\\ mb&a\end{array}} = \nOf{\begin{array}{c c}c&d\\ md&c\end{array}}
\end{equation}
