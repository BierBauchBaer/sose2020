\usepackage[utf8]{inputenc}
% Sonderzeichen
\usepackage{amssymb}
\usepackage{amsmath}
% Formeln, ...
\usepackage[german]{babel}
% Deutsch
\usepackage{cmbright}
% Font
\usepackage{graphicx}
% Includegraphics
\usepackage{fancyhdr}
% Pagestyle
\usepackage{titlesec}
% Titleformat
\usepackage[left=2cm,right=2cm,top=2.5cm,bottom=3cm]{geometry}
% Rand
\usepackage{xcolor}
\usepackage{color}
% Farbe in Tabellen und Plots
\usepackage{float}
% figures mit [h]
\usepackage{capt-of}
% Captions mit captionof{<figure/table>}{<Name>}
\usepackage{pgfplotstable}
% Tabellen aus .dat
\usepackage{titletoc}
% Für Inhaltsverzeichnis
\usepackage{mathtools}
\mathtoolsset{showonlyrefs}
\everymath{\displaystyle}

\titlecontents{section}[1.5em]{}{\bfseries\contentslabel{1.5em}}{\hspace*{-2.3em}}{\titlerule*{ }\bfseries\contentspage}

\titlecontents{subsection}[2.5em]{}{}{}{\titlerule*{.}\contentspage}

\titleformat{\section}{\normalfont}{\Large\bfseries\thesection.}{8pt}{\Large\bfseries}

\titleformat{\subsection}[runin]{}{}{0pt}{\large\bfseries}
% Sections betiteln

\pagestyle{fancy}
\rhead{\begin{tabular}{l r}
  Name & Matrikelnummer\\
  Lukas Hantzko     & 10011048
\end{tabular}}
\lhead{\begin{tabular}{l}
  Physikpraktikum\\
  {\bfseries Optische Bauelemente}
\end{tabular}}
% Header

\setlength\parindent{0pt}
