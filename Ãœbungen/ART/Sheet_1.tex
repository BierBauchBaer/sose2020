\documentclass[11pt,a4paper]{scrartcl}

\newcommand{\blatt}{1}
\newcommand{\fach}{General Relativity}
\usepackage[utf8]{inputenc}
% Sonderzeichen
\usepackage{amssymb}
\usepackage{amsmath}
\usepackage{amsfonts}
\usepackage{dsfont}
% Formeln, ...
\usepackage[german]{babel}
% Deutsch
\usepackage{cmbright}
% Font
\usepackage{graphicx}
% Includegraphics
\usepackage{fancyhdr}
% Pagestyle
\usepackage{titlesec}
% Titleformat
\usepackage[left=2cm,right=2cm,top=2.5cm,bottom=3cm]{geometry}
% Rand
\usepackage{xcolor}
\usepackage{color}
% Farbe in Tabellen und Plots
\usepackage{float}
% figures mit [h]
\usepackage{capt-of}
% Captions mit captionof{<figure/table>}{<Name>}
\usepackage{pgfplotstable}
% Tabellen aus .dat
\usepackage{mathtools}
\mathtoolsset{showonlyrefs}
\everymath{\displaystyle}

\titleformat{\section}{\normalfont}{\large\bfseries\blatt.\thesection}{8pt}{\normalfont}

\renewcommand{\thesubsection}{\alph{subsection}}
\titleformat{\subsection}{\normalfont}{(\thesubsection)}{8pt}{\normalfont}
% Sections betiteln

\pagestyle{fancy}
\rhead{\begin{tabular}{l r}
  Falk Trittschanke & 10011108\\
  Lukas Hantzko     & 10011048
\end{tabular}}
\lhead{\begin{tabular}{l}
  Hausübung \blatt\\
  {\bfseries \fach}
\end{tabular}}
% Header

\setlength\parindent{0pt}

% Symbole
\newcommand{\mE}[1]{\text{e}^{#1} }
\newcommand{\mI}{\text{i} }
\newcommand{\mQ}{\mathbb{Q} }
\newcommand{\mR}{\mathbb{R} }
\newcommand{\mC}{\mathscr{C}}
\newcommand{\mZ}{\mathbb{Z}}
\newcommand{\cC}{\mathbb{C}}
% Operatoren
\DeclareMathOperator{\tr}{Tr}
\newcommand*\Laplace{\mathop{}\!\mathbin\bigtriangleup}
\newcommand*\Nabla{\mathop{}\!\mathbin\bigtriangledown}
\newcommand*\DAlambert{\mathop{}\!\mathbin\Box}
% Klammern
\newcommand{\nOf}[1]{\left( #1 \right)}
\newcommand{\cOf}[1]{\left\{ #1 \right\}}
\newcommand{\eOf}[1]{\left[ #1 \right]}
\newcommand{\vOf}[1]{\left\lvert #1 \right\rvert}
\newcommand{\aOf}[1]{\left\langle #1 \right\rangle}
% Ableitungen
\newcommand{\dv}[2]{\frac{\text{d} #1}{\text{d} #2}}
% Besonderes
\newcommand{\scP}[2]{\aOf{#1 , #2}}


\begin{document}
	\section{UFF}
	The Person will be seeing a Force og $m_g\cdot g$ which is equal to $m_i\cdot a + F_{scale}$ meaning the Force, the scale and the person interact with. So the Force, the scale sees is $F_{scale} = - m_i\cdot a + m_g\cdot g$

	\section{energy conservation}
	The total momentum is given by $p = p_1 + p_2$. This is preserved, if $\pdv{p}{t} = 0$:
	\begin{align}
		\pdv{p_1}{t} = F_{1,2} & = - G m_{1,p} m_{2,a} \frac{x_1-x_2}{\vOf{x_1-x_2}^{3}} \\
		\pdv{p_2}{t} = F_{2,1} & = - G m_{2,p} m_{1,a} \frac{x_2-x_1}{\vOf{x_2-x_1}^{3}}
	\end{align}
	So that now we have equivalences:
	\begin{align}
		& \pdv{p}{t} = 0 \\
		\Leftrightarrow\quad & F_{1,2} + F_{2,1} = 0 \\
		\Leftrightarrow\quad & - G \frac{1}{\vOf{x_2-x_1}^3}\nOf{m_{1,p} m_{2,a} \nOf{x_1-x_2} + m_{2,p} m_{1,a} \nOf{x_2-x_1}} = 0 \\
		\Leftrightarrow\quad & m_{1,p}m_{2,a} = m_{2,p}m_{1,a} \\
		\Leftrightarrow\quad & \frac{m_{1,p}}{m_{1,a}} = \frac{m_{2,p}}{m_{2,a}} 
	\end{align}

	\section{gravitational field of a mass distribution}
	With the definition of $M$, we have the following:
	\begin{align}
		M & \mDef \lim\limits_{r\to\infty} \cOf{\frac{1}{4\pi G}\int\limits_{S^2\nOf{r}}\Nabla\phi\cdot \vN\dd{o}} \\
		& = \lim\limits_{r\to\infty} \cOf{\frac{1}{4\pi G}\int\limits_{B_0\nOf{r}}\Nabla\Nabla\phi\dd{v}} \\
		& = \lim\limits_{r\to\infty} \cOf{\frac{1}{4\pi G}\int\limits_{B_0\nOf{r}}\Laplace\phi\dd{v}} \\
		& = \lim\limits_{r\to\infty} \cOf{\frac{1}{4\pi G}\int\limits_{B_0\nOf{r}}4\pi G \rho \dd{v}} \\
		& = \lim\limits_{r\to\infty} \cOf{\int\limits_{B_0\nOf{r}}\rho\dd{v}} \\
		& = \int\limits_{\mR} \rho \dd{v}
	\end{align}
	which needed to be shown. \\
	To show the next identity, one just has to write out and use the definitions.
	\begin{align}
		f\indices{_a} & = -\Nabla\indices{^b}t\indices{_a_b} \\
		& = -\Nabla\indices{^b}\frac{1}{4\pi G}\nOf{\Nabla\indices{_a}\phi\Nabla\indices{_b}\phi - \frac{1}{2}\delta\indices{_a_b}\Nabla\indices{_c}\phi\Nabla\indices{^c}\phi} \\
		& = \frac{1}{4\pi G}\nOf{\Nabla\indices{^b}\Nabla\indices{_a}\Phi\Nabla\indices{_b}\phi+\Nabla\indices{_a}\phi\Nabla\indices{^b}\Nabla\indices{_b}\phi-\frac12\Nabla\indices{^a}\Nabla\indices{_c}\phi\Nabla\indices{^c}\phi-\frac12\Nabla\indices{_c}\phi\Nabla\indices{_a}\Nabla\indices{^c}\phi} \\
		& = \frac{1}{4\pi G}\nOf{\Nabla\indices{_a}\phi\Nabla\indices{^b}\Nabla\indices{_b}\phi} \\
		& = \frac{1}{4\pi G}\nOf{\Nabla\indices{_a}\phi\Laplace\phi} \\
		& = \rho\Nabla\indices{_a}
	\end{align}
	the result, as was expected. The tensor $t$ is symmetric, because all tensors in the definition of $t$ are symmetric. \\
	The total force vanishes because of the gaussian integral theorem:
	\begin{equation}
		\int\limits_{\mR} \Nabla\indices{^b}t\indices{_a_b} \dd{v} = \int\limits_{\partial\mR} t\indices{_a_b}n\indices{^b}\dd{a} = 0
	\end{equation}
	Same for the total torque:
	\begin{equation}
		\int\limits_{\mR} \Nabla\indices{^b}t\indices{_a_b}x\indices{^a}\dd{v} = \int\limits_{\partial\mR} t\indices{_a_b}x\indices{^a}n\indices{^b}\dd{a} = 0
	\end{equation}
	where in both cases the point being is, that $\partial\mR$, the border of $\mR$ is empty.
\end{document}
