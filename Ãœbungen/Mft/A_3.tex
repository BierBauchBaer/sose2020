\section{Einige Karten}

\subsection{Sei $U \subset \mR^n$ eine beliebige offene Menge in der Standardtopologie. Statten Sie nun $U$ mit einer $n$-dimensionalen Karte aus.}
\begin{equation}
	f : \mR^n \to \mR^n; x \mapsto x
\end{equation}
Ist eine Karte von $U$, da Bild und Urbild $U$ jeweils offen sind, und die Identität bijektiv ist.

\subsection{Ist diese Konstruktion auch für beliebige abgeschlossene Mengen des $\mR^n$ möglich?}
Nein, da die Abbildung dann nicht bijektiv von einer offenen Menge in eine offene Menge abbildet.

\subsection{Stereographische Projektion ist Karte der $S^n$. Weitere Karten für $2\nOf{n+1}$ Hemisphären $U_{i,\pm}$ für $i = 1,\dots,n+1$. Braucht man alle Hemisphären für Überdeckung? Kartenwechsel $\rightarrow$ $\mC^1$-Atlas?}
Nein, da man die stereographische Projektion folgendermaßen definieren kann, so dass man also nur zwei Karten braucht.
