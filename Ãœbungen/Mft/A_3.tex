\section{Einige Karten}

\subsection{Sei $U \subset \mR^n$ eine beliebige offene Menge in der Standardtopologie. Statten Sie nun $U$ mit einer $n$-dimensionalen Karte aus.}
\begin{equation}
	f : \mR^n \to \mR^n; x \mapsto x
\end{equation}
Ist eine Karte von $U$, da Bild und Urbild $U$ jeweils offen sind, und die Identität bijektiv ist.

\subsection{Ist diese Konstruktion auch für beliebige abgeschlossene Mengen des $\mR^n$ möglich?}
Nein, da diese Abbildung dann nicht bijektiv von einer offenen Menge in eine offene Menge abbildet. Man könnte 

\subsection{Stereographische Projektion ist Karte der $S^n$. Weitere Karten für $2\nOf{n+1}$ Hemisphären $U_{i,\pm}$ für $i = 1,\dots,n+1$. Braucht man alle Hemisphären für Überdeckung? Kartenwechsel $\rightarrow$ $\mC^1$-Atlas?}
Für die Hemisphären $U_{1,\pm}$ gibt es zum Beispiel die Polarkoordinaten.
Man braucht nicht alle Karten, da man auch eine Karte mit größerem Definitionsbereich nehmen könnte, in der dann zum Beispiel nur ein Pol fehlt. Nimmt man dann die andere Hemisphäre, so hat man die ganze $S^n$ überdeckt.