\newcounter{topGrd}
\setcounter{topGrd}{2}
\section{Topologische Grundlagen \roman{topGrd}}
\subsection{Seien $M$, $N$ topologische Räume und $f:M\to N$ eine stetige surjektive Abbildung. Falls $M$ zusammenhängend ist, dann ist auch $N$ zusammenhängend.}


\subsection{Gilt die Umkehrung der obigen Aussage?}


\subsection{Welche Aussage aus Analysis I wird durch (a) verallgemeinert?}


\subsection{Sei $M$ wegzusammenhängend. Zeigen Sie, dass $M$ auch zusammenhängend ist.} 
Sei $M$ nicht zusammenhängend.
Dann existieren $A_1, A_2 \in O_M$, die disjunkt sind und $A_1\cup A_2 = M$ erfüllen. Sei nun $\gamma$ ein beliebiger stetiger Weg von $a\in A_1$ nach $b\in A_2$, mit der Eigenschaft, dass $\gamma\nOf{t} \in M \,\forall\, t\in \eOf{0,1}$. Da $M$ offen ist, existiert für jeden Punkt auf $\gamma$ eine offene Umgebung $U\nOf{\gamma\nOf{t}}$ die in $M$ enthalten ist. Vereinigen wir nun, so muss folgende Menge in $O_M$ enthalten sein:
\begin{equation}
	A \mDef \bigcup\limits_{t\in\eOf{0,1}} U\nOf{\gamma\nOf{t}} \in O_M
\end{equation}
Nun gilt:
\begin{align}
	A \cap A_1 & \neq \emptyset \\
	A \cap A_2 & \neq \emptyset \\
	A_1 \cup A_2 & = M \\
	A_1 \cap A_2 & = \emptyset \\
	A \subset M \\