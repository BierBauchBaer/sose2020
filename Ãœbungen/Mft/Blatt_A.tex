\documentclass[11pt,a4paper]{scrartcl}

\newcommand{\blatt}{A}
\newcommand{\fach}{Mannigfaltigkeiten}
\usepackage[utf8]{inputenc}
% Sonderzeichen
\usepackage{amssymb}
\usepackage{amsmath}
\usepackage{amsfonts}
\usepackage{dsfont}
% Formeln, ...
\usepackage[german]{babel}
% Deutsch
\usepackage{cmbright}
% Font
\usepackage{graphicx}
% Includegraphics
\usepackage{fancyhdr}
% Pagestyle
\usepackage{titlesec}
% Titleformat
\usepackage[left=2cm,right=2cm,top=2.5cm,bottom=3cm]{geometry}
% Rand
\usepackage{xcolor}
\usepackage{color}
% Farbe in Tabellen und Plots
\usepackage{float}
% figures mit [h]
\usepackage{capt-of}
% Captions mit captionof{<figure/table>}{<Name>}
\usepackage{pgfplotstable}
% Tabellen aus .dat
\usepackage{mathtools}
\mathtoolsset{showonlyrefs}
\everymath{\displaystyle}

\titleformat{\section}{\normalfont}{\large\bfseries\blatt.\thesection}{8pt}{\normalfont}

\renewcommand{\thesubsection}{\alph{subsection}}
\titleformat{\subsection}{\normalfont}{(\thesubsection)}{8pt}{\normalfont}
% Sections betiteln

\pagestyle{fancy}
\rhead{\begin{tabular}{l r}
  Falk Trittschanke & 10011108\\
  Lukas Hantzko     & 10011048
\end{tabular}}
\lhead{\begin{tabular}{l}
  Hausübung \blatt\\
  {\bfseries \fach}
\end{tabular}}
% Header

\setlength\parindent{0pt}

% Symbole
\newcommand{\mE}[1]{\text{e}^{#1} }
\newcommand{\mI}{\text{i} }
\newcommand{\mQ}{\mathbb{Q} }
\newcommand{\mR}{\mathbb{R} }
\newcommand{\mC}{\mathscr{C}}
\newcommand{\mZ}{\mathbb{Z}}
\newcommand{\cC}{\mathbb{C}}
% Operatoren
\DeclareMathOperator{\tr}{Tr}
\newcommand*\Laplace{\mathop{}\!\mathbin\bigtriangleup}
\newcommand*\Nabla{\mathop{}\!\mathbin\bigtriangledown}
\newcommand*\DAlambert{\mathop{}\!\mathbin\Box}
% Klammern
\newcommand{\nOf}[1]{\left( #1 \right)}
\newcommand{\cOf}[1]{\left\{ #1 \right\}}
\newcommand{\eOf}[1]{\left[ #1 \right]}
\newcommand{\vOf}[1]{\left\lvert #1 \right\rvert}
\newcommand{\aOf}[1]{\left\langle #1 \right\rangle}
% Ableitungen
\newcommand{\dv}[2]{\frac{\text{d} #1}{\text{d} #2}}
% Besonderes
\newcommand{\scP}[2]{\aOf{#1 , #2}}


\begin{document}
	\section{Topologische Grundlagen}
	Seien $\nOf{M,O_M}$ und $\nOf{N,O_N}$ zwei Hausdorffräume.

	\subsection{Sei $\nOf{K,O_K}$ ein kompakter Raum. Zeigen Sie, dass eine abgeschlossene Menge $A\subset K$ auch kompakt ist.}
	Sei eine beliebige offene Überdeckung von $A$ gegeben. Dann ist 
 
	\subsection{Gilt dies auch für beliebige offene Teilmengen?}
	In $\mR$ ist eine offenes Intervall nicht kompakt, zum Beispiel $\nOf{0,1}$ ist eine offene Teilmenge einer kompakten Menge, zum Beispiel $\eOf{0,1}$, die mit der üblichen Topologie ein kompakter topologischer Raum ist.

	\subsection{Zeigen Sie, dass in der Teilraumtopologie Unterräume von $M$ Hausdorffräume sind.}
	Aufgrund der Teilraumtopologie sind Unterräume $U$ auch topologische Räume. Bleibt zu zeigen, dass in Unterräumen auch die Hausdorffeigenschaft erfüllt ist. Nehmen wir also zwei Punkte $x,y \in U$. Diese besitzen, da $U\subset M$ ein Hausdorffraum ist zwei offene Umgebungen $V_x, V_y \in O_M$, so dass gilt: $V_x \cap V_y = \emptyset$ (disjunkt). Nun definieren wir $U_x, U_y \subset U$ als $U_{\cdot} = V_{\cdot} \cap U$, also der Schnitt mit $U$. Diese sind in der Teilraumtopologie $U_x,U_y \in O_U = \sDf{A \cap U}{A \in O_M}$, da $V_x, V_y \in O_M$. Außerdem sind sie disjunkt, da $U_x \cap U_y = V_x \cap U \cap V_y \cap U = V_x \cap V_y = \emptyset$ nach Vorraussetzung. Damit sind diese Punkte durch disjunkte offene Umgebungen getrennt. 
% Fertig
	\subsection{Sei $f: M \to N$ stetig und $K \subset M$ überdeckungskompakt. Dann ist $f\nOf{K} \subset N$ ebenfalls ü-kompakt.}
	Sei eine offene Überdeckung von $f\nOf{K}$ gegeben:
	\begin{equation}
		f\nOf{K} \subset \Cup_{i\in I} A_i \subset N
	\end{equation}
	Dann ist folgendes auch eine offene Überdeckung, da $f$ stetig:
	\begin{equation}
		K \subset \Cup_{i\in I} f^{-1}\nOf{A_i} \subset M
	\end{equation}
	Dann existiert eine endliche Teilüberdeckung (K überdeckungskompakt)
	\begin{equation}
		K \subset \Cup_{i = 1}^n f^{-1}\nOf{A_i} \subset M
	\end{equation}
	Das Bild davon ist dann eine endliche Teilüberdeckung von $f\nOf{K}$:
	\begin{equation}
		f\nOf{K} \subset \Cup_{i = 1}^n A_i \subset N
	\end{equation}
% Fertig
	\section{Einsteinsche Summenkonvention}

	\subsection{Formulieren Sie mit der Summenkonvention die folgenden Begriffe der Linearen Algebra:}

	\subsubsection{Standardskalarprodukt des $\mR^n$}
	\begin{equation}
		v \cdot w = v_i w^i = \sum\limits_{i=1}^{n} v_i w_i
	\end{equation}

	\subsubsection{Matrix-Vektor-Produkt}
	\begin{equation}
		b = A v \quad b^i = A\indices{^i_j} v^j = \sum\limits_{j=1}^n A\indices{^i_j} v^j
	\end{equation}

	\subsubsection{Matrizenmultiplikation}
	\begin{equation}
		C = A B \quad C\indices{^i_k} = A\indices{^i_j} B\indices{^j_k} = \sum\limits_{j=1}^n A\indices{^i_j} B\indices{^j_k}
	\end{equation}

	\subsubsection{Spur einer Matrix}
	\begin{equation}
		\tr\nOf{A} = A\indices{^j_j} = \sum\limits_{j=1}^n A\indices{^j_j}
	\end{equation}

	\subsubsection{Transponieren einer Matrix}
	\begin{equation}
		B = A^T \quad B\indices{^i_j} = A\indices{^j_i}
	\end{equation}

	\subsection{Das Levi-Civita-Symbol}
	Wir nehmen an:
	\begin{equation}
	x = (x^1,x^2,x^3) \quad y = (y^1,y^2,y^3) \quad z = (z^1,z^2,z^3)
	\end{equation}
	Behauptung: Es wird das Kreuzprodukt $z = x \times y$ berechnet.\\
	Begründung: Komponentenweise nachrechnen:
	\begin{align}
		z^1 & = \sum\limits_{i=1}^3\sum\limits_{j=1}^3 \epsilon\indices{_i_j^1} x^i y^j \\
		& = \sum\limits_{i=1}^3 \nOf{\epsilon\indices{_i_2^1} x^iy^2 + \epsilon\indices{_i_3^1} x^i y^3} \\
		& = \epsilon\indices{_3_2^1} x^3 y^2 + \epsilon\indices{_2_3^1} x^2 y^3 \\
		& = x^2 y^3 - x^3 y^2
	\end{align}
	Analog für die anderen Komponenten (zyklische Vertauschung der Indizes)
	\begin{equation}
		z = (x^2 y^3 - x^3 y^2, x^3 y^1 - x^1 y^3, x^1 y^2 - x^2 y^1)
	\end{equation}

	\subsection{Beweisen Sie für $f, g \in \mC^1\nOf{\mR,\mR^n}$, dass}
	\begin{equation}
		\dv{}{t} \scP{f\nOf{t}}{g\nOf{t}} = \scP{\dv{}{t}f\nOf{t}}{g\nOf{t}} + \scP{f\nOf{t}}{\dv{}{t}g\nOf{t}}
	\end{equation}
	\begin{align}
		\dv{}{t} \scP{f\nOf{t}}{g\nOf{t}} & = \dv{}{t} f^i\nOf{t} g^i\nOf{t} \\
		& = \sum\limits_{i=1}^n \dv{}{t} f^i\nOf{t} g^i\nOf{t} \\
		& = \sum\limits_{i=1}^n \dv{f^i\nOf{t}}{t} g^i\nOf{t} + f^i\nOf{t} \dv{g^i\nOf{t}}{t} \\
		& = \sum\limits_{i=1}^n \dv{f^i\nOf{t}}{t} g^i\nOf{t} + \sum\limits_{i=1}^n \dv{g^i\nOf{t}}{t} f^i\nOf{t} \\
		& = \scP{\dv{}{t}f\nOf{t}}{g\nOf{t}} + \scP{f\nOf{t}}{\dv{}{t}g\nOf{t}}
	\end{align}

	\section{Einige Karten}

	\subsection{Sei $U \subset \mR^n$ eine beliebige offene Menge in der Standardtopologie. Statten Sie nun $U$ mit einer $n$-dimensionalen Karte aus.}
	\begin{equation}
		f : \mR^n \to \mR^n; x \mapsto x
	\end{equation}
	Ist eine Karte von $U$.

	\subsection{Ist diese Konstruktion auch für beliebige abgeschlossene Mengen des $\mR^n$ möglich?}
	Nein, da die Abbildung dann nicht bijektiv von einer offenen Menge in eine offene Menge abbildet.

	\subsection{Stereographische Projektion der $S^n$ Karte. Weitere Karten für $2\nOf{n+1}$ Hemisphären $U_{i,\pm}$ für $i = 1,\dots,n+1$. Alle Hemisphären für Überdeckung? Kartenwechsel $\rightarrow$ $\mC^1$-Atlas?}

\end{document}
