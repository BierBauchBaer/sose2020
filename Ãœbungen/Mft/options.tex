\usepackage[utf8]{inputenc}
% Sonderzeichen
\usepackage{amssymb}
\usepackage{amsmath}
\usepackage{amsfonts}
\usepackage{dsfont}
\usepackage{mathrsfs}
% Formeln, ...
\usepackage[german]{babel}
% Deutsch
\usepackage{cmbright}
% Font
\usepackage{graphicx}
% Includegraphics
\usepackage{fancyhdr}
% Pagestyle
\usepackage{titlesec}
% Titleformat
\usepackage[left=2cm,right=2cm,top=2.5cm,bottom=3cm]{geometry}
% Rand
\usepackage{xcolor}
\usepackage{color}
% Farbe in Tabellen und Plots
\usepackage{float}
% figures mit [h]
\usepackage{capt-of}
% Captions mit captionof{<figure/table>}{<Name>}
\usepackage{pgfplotstable}
% Tabellen aus .dat
\usepackage{mathtools}
\mathtoolsset{showonlyrefs}
\everymath{\displaystyle}
\usepackage{tensor}

\titleformat{\section}{\normalfont}{\large\bfseries Aufgabe \blatt.\thesection}{8pt}{\textit}

\renewcommand{\thesubsection}{\alph{subsection}}
\titleformat{\subsection}{\normalfont}{(\thesubsection)}{8pt}{\normalfont}

\renewcommand{\thesubsubsection}{\arabic{subsubsection}}
\titleformat{\subsubsection}{\normalfont}{(\thesubsubsection)}{8pt}{\normalfont}
% Sections betiteln

\pagestyle{fancy}
\rhead{\begin{tabular}{r}
  Lukas Hantzko \\
	10011048
\end{tabular}}
\lhead{\begin{tabular}{l}
  Hausübung \blatt\\
  {\bfseries \fach}
\end{tabular}}
% Header

\setlength\parindent{0pt}
