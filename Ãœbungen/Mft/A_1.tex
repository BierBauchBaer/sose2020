\section{Topologische Grundlagen}
Seien $\nOf{M,O_M}$ und $\nOf{N,O_N}$ zwei Hausdorffräume.

\subsection{Sei $\nOf{K,O_K}$ ein kompakter Raum. Zeigen Sie, dass eine abgeschlossene Menge $A\subset K$ auch kompakt ist.}
Sei eine beliebige offene Überdeckung $B_i\in O_K$, $i\in I$ von $A$ gegeben. $A^C \in O_K$
Dann ist $C_i \mDef A^C\cup B_i$ eine offene Überdeckung von $K$, nach Vorraussetzung ist $K$ kompakt, also existiert eine endliche Teilüberdeckung ($I_n\subset I : \vOf{I_n} = n$):
\begin{equation}
	K \subset \bigcup\limits_{i\in I_n} C_i
\end{equation}
$B_i = C_i\cap A$, $i\in I_n$ ist dann eine endliche Teilüberdeckung von $B_i$:
\begin{equation}
	A\subset \bigcup\limits_{i\in I_n} B_i
\end{equation}

\subsection{Gilt dies auch für beliebige offene Teilmengen?}
In $\mR$ ist ein offenes Intervall nicht kompakt, zum Beispiel $\nOf{0,1}$ ist eine offene Teilmenge einer kompakten Menge, zum Beispiel das Intervall $\eOf{0,1}$, die mit der üblichen Topologie ein kompakter topologischer Raum ist. Im Beweis sind die $C_i$ nicht notwendigerweise offen, weshalb nicht unbedingt eine endliche Teilüberdeckung existieren muss.

\subsection{Zeigen Sie, dass in der Teilraumtopologie Unterräume von $M$ Hausdorffräume sind.}
Aufgrund der Teilraumtopologie sind Unterräume $U$ auch topologische Räume.\\
Bleibt zu zeigen, dass in Unterräumen auch die Hausdorffeigenschaft erfüllt ist. Nehmen wir also zwei Punkte $x,y \in U$. Diese besitzen, da $U\subset M$ ein Hausdorffraum ist zwei offene Umgebungen $V_x, V_y \in O_M$, so dass gilt: $V_x \cap V_y = \emptyset$ (disjunkt).\\
Nun definieren wir $U_x, U_y \subset U$ über $U_{\cdot} = V_{\cdot} \cap U$, also der Schnitt mit $U$. Diese sind in der Teilraumtopologie $U_x,U_y \in O_U = \sDf{A \cap U}{A \in O_M}$, da $V_x, V_y \in O_M$. Außerdem sind sie disjunkt, da $U_x \cap U_y = V_x \cap U \cap V_y \cap U = V_x \cap V_y = \emptyset$ nach Vorraussetzung. Damit sind diese Punkte durch disjunkte offene Umgebungen getrennt.

\subsection{Sei $f: M \to N$ stetig und $K \subset M$ überdeckungskompakt (ük). Dann ist $f\nOf{K} \subset N$ ebenfalls ük.}
Sei eine offene Überdeckung von $f\nOf{K}$ gegeben:
\begin{equation}
	f\nOf{K} \subset \bigcup\limits_{i\in I} A_i \subset N
\end{equation}
Dann ist folgendes auch eine offene Überdeckung, da $f$ stetig:
\begin{equation}
	K \subset \bigcup\limits_{i\in I} f^{-1}\nOf{A_i} \subset M
\end{equation}
Dann existiert eine endliche Teilüberdeckung ($I_n\subset I : \vOf{I_n} = n$) (K überdeckungskompakt)
\begin{equation}
	K \subset \bigcup\limits_{I_n} f^{-1}\nOf{A_i} \subset M
\end{equation}
Das Bild davon ist dann eine endliche Teilüberdeckung der ursprünglichen Überdeckung von $f\nOf{K}$:
\begin{equation}
	f\nOf{K} \subset \bigcup\limits_{I_n} A_i \subset N
\end{equation}
