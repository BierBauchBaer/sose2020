\section{Komplexe Zahlen}
	Es gilt, alle Lösungen der folgenden Gleichungen über $\mC$ zu finden.
	\subsection{$z^2-10 z + 34= 0$}
	Polynom zweiten Grades ermöglicht p-q-Formel:
	\begin{equation}
		z_\pm = 5 \pm \sqrt{25-34} = 5 \pm \mI \sqrt{9} = 5 \pm \mI 3
	\end{equation}

	\subsection{$z^3-3 z\bar{z} = - 2$}
	Bedingungen an Nullstellen: $z^3 \in \mR$, da $-3z\bar{z}+2\in \mR$\\
	Andererseits für die Beträge ($z=r\mE{\mI\varphi}$): $r^3-3r^2+2=0$. Dies ist ein (reelles) Polynom, und die Lösung $r=1$ kann leicht überprüft werden.\\
	Per Polynomdivision lässt sich ein Linearfaktor $r-1$ abspalten: $r^3-3r^2+2 = (r-1)(r^2-2r-2)$.\\
	Hier lassen sich die anderen Nullstellen per p-q-Formel bestimmen. $r = 1 \pm \sqrt{1+2} = 1\pm\sqrt{3}$\\
	Fall: $r = 1$, dann muss $\mE{3\mI\varphi} = 1$ sein, und es ergeben sich drei Nullstellen: 
	\begin{gather}
		\mE{\mI\frac{2\pi}{3}} \\
		\mE{\mI 2\frac{2\pi}{3}}\\
		\mE{\mI 3\frac{2\pi}{3}}=1
	\end{gather}
	Fall: $r = (1 - \sqrt{3})$, dann muss $r^3\mE{3\mI\varphi} - 3 r^2 + 2 = 0$ sein, also:
	\begin{align}
		(1 - 3\sqrt{3} + 9 - 3\sqrt{3})\mE{3\mI\varphi} - 3 + 6 \sqrt{3} - 9 +2 & = 0 \\
		(10-6\sqrt{3})(\mE{3\mI\varphi}-1) & = 0
	\end{align}
	Dafür ergibt sich $\mE{\3mI\varphi} = 1$, also drei Nullstellen
	\begin{gather}
		(1 - \sqrt{3})\mE{\mI\frac{2\pi}{3}} \\
		(1 - \sqrt{3})\mE{\mI 2\frac{2\pi}{3}}\\
		(1 - \sqrt{3})\mE{\mI 3\frac{2\pi}{3}}
	\end{gather}
	Fall: $r = (1 + \sqrt{3})$, dann gilt:
	\begin{align}
		(1 + 3\sqrt{3} + 9 + 3\sqrt{3})\mE{\mI\varphi} - 3 - 6 \sqrt{3} - 9 +2 & = 0 \\
		(10+6\sqrt{3})(\mE{\mI\varphi}-1) & = 0
	\end{align}
	Das ergibt analog:
	\begin{gather}
		(1 + \sqrt{3})\mE{\mI\frac{2\pi}{3}} \\
		(1 + \sqrt{3})\mE{\mI 2\frac{2\pi}{3}}\\
		(1 + \sqrt{3})\mE{\mI 3\frac{2\pi}{3}}
	\end{gather}

	\subsection{$z^4 = 1 - \mI \sqrt{3}$}
	Hier lässt sich auch wieder der Betrag und das Argument getrennt betrachten.
	\begin{equation}
		r^4 = \vOf{1-\mI\sqrt{3}} = \sqrt{1^2 + (-\sqrt{3})^2} = 2 \quad 4\varphi = \arctan\nOf{-\sqrt{3}} = -\frac{\pi}{3} \mod\nOf{2\pi}
	\end{equation}
	Damit ergeben sich vier Lösungen (:
	\begin{gather}
		\sqrt[4]{2} \mE{-\mI\frac{2\pi}{24}} \\
		\sqrt[4]{2} \mE{-\mI\frac{8\pi}{24}} \\
		\sqrt[4]{2} \mE{-\mI\frac{14\pi}{24}} \\
		\sqrt[4]{2} \mE{-\mI\frac{20\pi}{24}} \\
	\end{gather}