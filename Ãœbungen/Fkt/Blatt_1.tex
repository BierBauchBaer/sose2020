\documentclass[11pt,a4paper]{scrartcl}

% Allgemein
\newcommand{\blatt}{1}
\newcommand{\fach}{Funktionentheorie}
\usepackage[utf8]{inputenc}
% Sonderzeichen
\usepackage{amssymb}
\usepackage{amsmath}
\usepackage{amsfonts}
\usepackage{dsfont}
% Formeln, ...
\usepackage[german]{babel}
% Deutsch
\usepackage{cmbright}
% Font
\usepackage{graphicx}
% Includegraphics
\usepackage{fancyhdr}
% Pagestyle
\usepackage{titlesec}
% Titleformat
\usepackage[left=2cm,right=2cm,top=2.5cm,bottom=3cm]{geometry}
% Rand
\usepackage{xcolor}
\usepackage{color}
% Farbe in Tabellen und Plots
\usepackage{float}
% figures mit [h]
\usepackage{capt-of}
% Captions mit captionof{<figure/table>}{<Name>}
\usepackage{pgfplotstable}
% Tabellen aus .dat
\usepackage{mathtools}
\mathtoolsset{showonlyrefs}
\everymath{\displaystyle}

\titleformat{\section}{\normalfont}{\large\bfseries\blatt.\thesection}{8pt}{\normalfont}

\renewcommand{\thesubsection}{\alph{subsection}}
\titleformat{\subsection}{\normalfont}{(\thesubsection)}{8pt}{\normalfont}
% Sections betiteln

\pagestyle{fancy}
\rhead{\begin{tabular}{l r}
  Falk Trittschanke & 10011108\\
  Lukas Hantzko     & 10011048
\end{tabular}}
\lhead{\begin{tabular}{l}
  Hausübung \blatt\\
  {\bfseries \fach}
\end{tabular}}
% Header

\setlength\parindent{0pt}

% Symbole
\newcommand{\mE}[1]{\text{e}^{#1} }
\newcommand{\mI}{\text{i} }
\newcommand{\mQ}{\mathbb{Q} }
\newcommand{\mR}{\mathbb{R} }
\newcommand{\mC}{\mathscr{C}}
\newcommand{\mZ}{\mathbb{Z}}
\newcommand{\cC}{\mathbb{C}}
% Operatoren
\DeclareMathOperator{\tr}{Tr}
\newcommand*\Laplace{\mathop{}\!\mathbin\bigtriangleup}
\newcommand*\Nabla{\mathop{}\!\mathbin\bigtriangledown}
\newcommand*\DAlambert{\mathop{}\!\mathbin\Box}
% Klammern
\newcommand{\nOf}[1]{\left( #1 \right)}
\newcommand{\cOf}[1]{\left\{ #1 \right\}}
\newcommand{\eOf}[1]{\left[ #1 \right]}
\newcommand{\vOf}[1]{\left\lvert #1 \right\rvert}
\newcommand{\aOf}[1]{\left\langle #1 \right\rangle}
% Ableitungen
\newcommand{\dv}[2]{\frac{\text{d} #1}{\text{d} #2}}
% Besonderes
\newcommand{\scP}[2]{\aOf{#1 , #2}}


\begin{document}
	\section{Komplexe Zahlen}
	Es gilt, alle Lösungen der folgenden Gleichungen über $\mC$ zu finden.
	\subsection{$z^2-10 z + 34= 0$}
	Polynom zweiten Grades ermöglicht p-q-Formel:
	\begin{equation}
		z_\pm = 5 \pm \sqrt{25-34} = 5 \pm \mI \sqrt{9} = 5 \pm \mI 3
	\end{equation}

	\subsection{$z^3-3 z\bar{z} = - 2$}
	Bedingungen an Nullstellen: $z^3 \in \mR$, da $-3z\bar{z}+2\in \mR$\\
	Andererseits für die Beträge ($z=r\mE{\mI\varphi}$): $r^3-3r^2+2=0$. Dies ist ein (reelles) Polynom, und die Lösung $r=1$ kann leicht überprüft werden.\\
	Per Polynomdivision lässt sich ein Linearfaktor $r-1$ abspalten: $r^3-3r^2+2 = (r-1)(r^2-2r-2)$.\\
	Hier lassen sich die anderen Nullstellen per p-q-Formel bestimmen. $r = 1 \pm \sqrt{1+2} = 1\pm\sqrt{3}$\\
	Fall: $r = 1$, dann muss $\mE{\mI\varphi} = 1$ sein, und es ergeben sich drei Nullstellen: 
	\begin{gather}
		\mE{\mI\frac{2\pi}{3}} \\
		\mE{\mI 2\frac{2\pi}{3}}\\
		\mE{\mI 3\frac{2\pi}{3}}=1
	\end{gather}
	Fall: $r = (1 - \sqrt{3})$, dann muss $r^3\mE{\mI\varphi} - 3 r^2 + 2 = 0$ sein, also:
	\begin{align}
		(1 - 3\sqrt{3} + 9 - 3\sqrt{3})\mE{\mI\varphi} - 3 + 6 \sqrt{3} - 9 +2 & = 0 \\
		(10-6\sqrt{3})(\mE{\mI\varphi}-1) & = 0
	\end{align}
	Dafür ergibt sich $\mE{\mI\varphi} = 1$, also drei Nullstellen
	\begin{gather}
		(1 - \sqrt{3})\mE{\mI\frac{2\pi}{3}} \\
		(1 - \sqrt{3})\mE{\mI 2\frac{2\pi}{3}}\\
		(1 - \sqrt{3})\mE{\mI 3\frac{2\pi}{3}}
	\end{gather}
	Fall: $r = (1 + \sqrt{3})$, dann gilt:
	\begin{align}
		(1 + 3\sqrt{3} + 9 + 3\sqrt{3})\mE{\mI\varphi} - 3 - 6 \sqrt{3} - 9 +2 & = 0 \\
		(10+6\sqrt{3})(\mE{\mI\varphi}-1) & = 0
	\end{align}
	Das ergibt analog:
	\begin{gather}
		(1 - \sqrt{3})\mE{\mI\frac{2\pi}{3}} \\
		(1 - \sqrt{3})\mE{\mI 2\frac{2\pi}{3}}\\
		(1 - \sqrt{3})\mE{\mI 3\frac{2\pi}{3}}
	\end{gather}

	\subsection{$z^4 = 1 - \mI \sqrt{3}$}
	Hier lässt sich auch wieder der Betrag und das Argument getrennt betrachten.
	\begin{equation}
		r^4 = \vOf{1-\mI\sqrt{3}} = \sqrt{1^2 + (-\sqrt{3})^2} = \sqrt{2} \quad 4\varphi = \arctan\nOf{-\sqrt{3}} = -\frac{\pi}{3} \mod\nOf{2\pi}
	\end{equation}
	Damit ergeben sich vier Lösungen (:
	\begin{gather}
		\sqrt[8]{2} \mE{-\mI\frac{2\pi}{24}} \\
		\sqrt[8]{2} \mE{-\mI\frac{8\pi}{24}} \\
		\sqrt[8]{2} \mE{-\mI\frac{14\pi}{24}} \\
		\sqrt[8]{2} \mE{-\mI\frac{20\pi}{24}} \\
	\end{gather}

\end{document}
