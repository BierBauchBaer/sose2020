\section{}
\subsection{}
    Sei $\Phi$ eine nicht identitäre Möbiustransformation, d.h.
    \begin{align}
		&\Phi (z)= 
		\begin{cases}
            \frac{az+b}{cz+d} &,cz+d \neq 0\\
            \infty & ,sonst
        \end{cases}
	\end{align}
	Für den Fall $z=\infty$ ergibt sich $\Phi(z)=\infty \overset{!}{=}z$ genau ein Fixpunkt.\\
	Sei nun also $z \in \mathbb{C}$:
	\begin{align}
		&\Phi(z)=\frac{az+b}{cz+d}\overset{!}{=}z\\
		&\Leftrightarrow z(cz+d)=az+b\\
		&\Leftrightarrow cz^2+(a+d)z-b=0
	\end{align}
	Fall $c=0, a+d=0$:
    \begin{align}
		& \flash \Rightarrow \text{kein Fixpunkt}
	\end{align}
    Der Fall $c=0, a+d \neq 0$ liefert genau einen Fixpunkt:
    \begin{align}
		& (a+d)z-b=0
        & \Leftrightarrow z= \frac{b}{a+d}
	\end{align}
	Für den Fall $c \neq 0$ liefert die pq-Formel maximal 2 Lösungen:
    \begin{align}
		& cz^2+(a+d)z-b=0
	\end{align}
	Damit gibt es maximal 2 Fixpunkte.
\subsection{}