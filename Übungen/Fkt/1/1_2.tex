
	\section{Aufgabe 2}
	
	\subsection{}
    $f(x,y)=x^3-3xy^2-2x+i(-y^3+3x^2y-2y) $ ist komplex differenzierbar in $x+iy$ gdw. die Jacobi-Matrix
    \begin{align*}
    \begin{pmatrix}
	\frac{d}{dx}x^3-3xy^2-2x & \frac{d}{dx}-y^3+3x^2y-2y\\ 
	\frac{d}{dy}x^3-3xy^2-2 & \frac{d}{dy}-y^3+3x^2y-2y
	\end{pmatrix}
	=\begin{pmatrix}
	3x^2-3y^2-2 & 6xy\\
    -6xy & -3y^2+3x^2-2
    \end{pmatrix}
    \end{align*}
    schiefsymmetrisch ist. Dies ist offensichtlich für alle $x+iy \in \mathbb{C}$ erfüllt.\\
    Damit ist f auf ganz $\mathbb{C}$ komplex differenzierbar.
    
    \subsection{}
    $g(x,y)=x^3+xy^2+i(y^3+3x^2y-2y) $ ist komplex differenzierbar in $x+iy$ gdw. die Jacobi-Matrix
    \begin{align*}
    \begin{pmatrix}
	\frac{d}{dx}x^3+xy^2 & \frac{d}{dx}-y^3+3x^2y-2y\\ 
	\frac{d}{dy}x^3+xy^2 & \frac{d}{dy}-y^3+3x^2y-2y
	\end{pmatrix}
	=\begin{pmatrix}
	3x^2+y^2 & 6xy\\
    2xy & 3y^2+3x^2-2
    \end{pmatrix}
    \end{align*}
    schiefsymmetrisch ist.\\
    Aus $-6xy=2xy$ folgt $x=0 \lor y=0$.\\
    Fall $x=0$: $y^2\stackrel{!}{=} 3y^2-2\Rightarrow y=\pm1$\\
    Fall $y=0$: $3x^2\stackrel{!}{=} 3x^2-2$ Widerspruch!\\
    $g$ erfüllt also nur auf $\{ \pm1 \}$ die CRD. Da jedoch $\{\pm1\}$ nicht offen ist, ist die größte Menge auf der $g$ komplex differenzierbar ist die leere Menge.
    \subsection{}
    $h(x,y)
    =sin(x+y^2-x^2+i(y-2xy))\\
    =\frac{1}{2i}(e^{i(x+y^2-x^2)-y+2xy}-e^{i(-x-y^2+x^2)+y-2xy})\\
    =\frac{-i}{2}(e^{-y+2xy}(cos(x+y^2-x^2)+isin(x+y^2-x^2))-e^{y-2xy}(cos(-x-y^2+x^2)+isin(-x-y^2+x^2)))\\
    =\frac{1}{2} e^{-y+2xy}sin(x+y^2-x^2)-\frac{1}{2}e^{y-2xy}sin(-x-y^2+x^2) +\frac{i}{2}(-e^{-y+2xy}cos(x+y^2-x^2)+e^{y-2xy}cos(-x-y^2+x^2)$\\
    Damit ist die Jacobimatrix gegeben durch die partiellen Ableitungen der Funktionen:
    \begin{align}
        u\nOf{x,y} & = \frac{1}{2}\nOf{\mE{-y+2xy}\sin\nOf{x+y^2-x^2}-\mE{y-2xy}\sin\nOf{-x-y^2+x^2}} \\
        v\nOf{x,y} & = \frac{1}{2}\nOf{-\mE{-y+2xy}\cos\nOf{x+y^2-x^2}+\mE{y-2xy}\cos\nOf{-x-y^2+x^2}}
    \end{align}
    \begin{equation}
        D_{\nOf{x,y}}\nOf{u,v} 
        = \nOf{\begin{array}{cc}\pdv{u}{x}&\pdv{u}{y}\\\pdv{v}{x}&\pdv{v}{y}\end{array}}
    \end{equation}
    \begin{align}
        \pdv{u}{x} & = \frac{1}{2}\nOf{2y\mE{-y+2xy}\sin\nOf{x+y^2-x^2}+\nOf{1-2x}\mE{-y+2xy}\cos\nOf{x+y^2-x^2}+ 2y\mE{y-2xy}\sin\nOf{-x-y^2+x^2}-\nOf{-1+2x}\mE{-y+2xy}\cos\nOf{-x-y^2+x^2}} \\
        & = y\sin\nOf{x+y^2-x^2}\nOf{\mE{-y+2xy}-\mE{y-2xy}}+\nOf{\frac12-x}\cos\nOf{x+y^2-x^2}\nOf{\mE{-y+2xy}+\mE{y-2xy}} \\
        \pdv{u}{y} & = \frac{1}{2}\nOf{\nOf{2x-1}\mE{-y+2xy}\sin\nOf{x+y^2-x^2}+2y\mE{-y+2xy}\cos\nOf{x+y^2-x^2}-\nOf{1-2x}\mE{y-2xy}\sin\nOf{-x-y^2+x^2}+2y\mE{y-2xy}\cos\nOf{-x-y^2+x^2}} \\
        & = y\cos\nOf{x+y^2-x^2}\nOf{\mE{y-2xy}+\mE{-y+2xy}}+\nOf{x-\frac12}\sin\nOf{x+y^2-x^2}\nOf{\mE{-y+2xy}-\mE{y-2xy}} \\
        \pdv{v}{x} & = \frac{1}{2}\nOf{-2y\mE{-y+2xy}\cos\nOf{x+y^2-x^2}+\nOf{1-2x}\mE{-y+2xy}\sin\nOf{x+y^2-x^2}-2y\mE{y-2xy}\cos\nOf{-x-y^2+x^2}-\nOf{2x-1}\mE{y-2xy}\sin\nOf{-x-y^2+x^2}} \\
        & = -y\cos\nOf{x+y^2-x^2}\nOf{\mE{-y+2xy}+\mE{y-2xy}}+\nOf{\frac12-x}\sin\nOf{x+y^2-x^2}\nOf{\mE{-y+2xy}-\mE{y-2xy}}\\
        \pdv{v}{y} & = \frac{1}{2}\nOf{\nOf{1-2x}\mE{-y+2xy}\cos\nOf{x+y^2-x^2}+2y\mE{-y+2xy}\sin\nOf{x+y^2-x^2}+\nOf{1-2x}\mE{y-2xy}\cos\nOf{-x-y^2+x^2}+2y\mE{y-2xy}\sin\nOf{-x-y^2+x^2}} \\
        & = -y\sin\nOf{x+y^2-x^2}\nOf{\mE{y-2xy}-\mE{-y+2xy}}+\nOf{\frac12-x}\cos\nOf{x+y^2-x^2}\nOf{\mE{y-2xy}+\mE{-y+2xy}}
    \end{align}
    Damit sind die Cauchy-Riemannschen Differentialgleichungen überall erfüllt, und die Funktion ist holomorph (komplex differenzierbar auf ganz $\mC$)