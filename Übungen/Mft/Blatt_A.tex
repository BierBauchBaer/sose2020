\documentclass[11pt,a4paper]{scrartcl}

\newcommand{\blatt}{A}
\newcommand{\fach}{Mannigfaltigkeiten}
\usepackage[utf8]{inputenc}
% Sonderzeichen
\usepackage{amssymb}
\usepackage{amsmath}
% Formeln, ...
\usepackage[german]{babel}
% Deutsch
\usepackage{cmbright}
% Font
\usepackage{graphicx}
% Includegraphics
\usepackage{fancyhdr}
% Pagestyle
\usepackage{titlesec}
% Titleformat
\usepackage[left=2cm,right=2cm,top=2.5cm,bottom=3cm]{geometry}
% Rand
\usepackage{xcolor}
\usepackage{color}
% Farbe in Tabellen und Plots
\usepackage{float}
% figures mit [h]
\usepackage{capt-of}
% Captions mit captionof{<figure/table>}{<Name>}
\usepackage{pgfplotstable}
% Tabellen aus .dat
\usepackage{titletoc}
% Für Inhaltsverzeichnis

\titlecontents{section}[1.5em]{}{\bfseries\contentslabel{1.5em}}{\hspace*{-2.3em}}{\titlerule*{ }\bfseries\contentspage}

\titlecontents{subsection}[2.5em]{}{}{}{\titlerule*{.}\contentspage}

\titleformat{\section}{\normalfont}{\Large\bfseries\thesection.}{8pt}{\Large\bfseries}

\titleformat{\subsection}[runin]{}{}{0pt}{\large\bfseries}
% Sections betiteln

\pagestyle{fancy}
\rhead{\begin{tabular}{l r}
  Name & Matrikelnummer\\
  Lukas Hantzko     & 10011048
\end{tabular}}
\lhead{\begin{tabular}{l}
  Physikpraktikum\\
  {\bfseries Optische Bauelemente}
\end{tabular}}
% Header

\setlength\parindent{0pt}
% Symbole
\newcommand{\mE}[1]{\text{e}^{#1}}
\newcommand{\mI}{\text{i}\,}
\newcommand{\cc}[1]{#1\!^{*}\!\,}
\newcommand{\mQ}{\mathbb{Q}}
\newcommand{\mR}{\mathbb{R}}
\newcommand{\mC}{\mathbb{C}}
\newcommand{\mN}{\mathbb{N}}
\newcommand{\mF}{\mathbb{F}}
\newcommand{\mZ}{\mathbb{Z}}
% Operatoren
\DeclareMathOperator{\gal}{Gal}
\DeclareMathOperator{\car}{char}
\DeclareMathOperator{\ord}{ord}
\DeclareMathOperator{\aut}{Aut}
\DeclareMathOperator{\fix}{Fix}
\DeclareMathOperator{\id}{id}
\DeclareMathOperator{\gl}{GL}
\DeclareMathOperator{\im}{Im}
% Klammern
\newcommand{\nOf}[1]{\left( #1 \right)}
\newcommand{\cOf}[1]{\left\{ #1 \right\}}
\newcommand{\eOf}[1]{\left[ #1 \right]}
\newcommand{\vOf}[1]{\left\lvert #1 \right\rvert}
\newcommand{\aOf}[1]{\left\langle #1 \right\rangle}


\newcommand{\sDf}[2]{\left\{ #1 \middle| #2 \right\}}
\newcommand{\mDef}{\vcentcolon =}
\newcommand{\isom}{\stackrel{\sim}{=}}
% BraKet
\newcommand{\bra}[1]{\left\langle #1 \right|}
\newcommand{\ket}[1]{\left| #1 \right\rangle}
\newcommand{\brKt}[2]{\left\langle #1 \middle| #2 \right\rangle}
\newcommand{\bkEx}[3]{\left\langle #1 \middle| #2 \middle| #3 \right\rangle}
% Operator
\newcommand{\op}[1]{\widehat{#1}}
\newcommand{\dg}[1]{\widehat{#1}\,^{\dagger}\!\,}


\begin{document}
	\section{Topologische Grundlagen}
	Seien $\nOf{M,O_M}$ und $\nOf{N,O_N}$ zwei Hausdorffräume.
	\subsection{Zeigen Sie, dass in der Teilraumtopologie abgeschlossene Teilmengen kompakter Mengen auch kompakt sind.}
	Kompakte Menge: Jede offene Überdeckung besitzt endliche Teilüberdeckung. Überdeckung im Teilraum $\rightarrow$ Überdeckung im Raum $\rightarrow$ endliche Überdeckung im Raum $\rightarrow$ Per Schnitt mit Menge: Endliche offene Überdeckung.
	\subsection{Gilt dies auch für beliebige offene Teilmengen?}
	\subsection{Zeigen Sie, dass in der Teilraumtopologie Unterräume von $M$ Hausdorffräume sind.}
	\subsection{Sei $f: M \to N$ stetig und $K \subset M$ überdeckungskompakt. Dann ist $f\nOf{K} \subset N$ ebenfalls ü-kompakt.}
	\section{Einsteinsche Summenkonvention}
	\subsection{Formulieren Sie mit der Summenkonvention die folgenden Begriffe der Linearen Algebra:}
	\subsubsection{Standardskalarprodukt des $\mR^n$}
	\begin{equation}
		v \cdot w = v_i w^i = \sum\limits_{i=1}^{n} v_i w_i
	\end{equation}
	\subsubsection{Matrix-Vektor-Produkt}
	\begin{equation}
		b = A v \quad b^i = A\indices{^i_j} v^j = \sum\limits_{j=1}^n A\indices{^i_j} v^j
	\end{equation}
	\subsubsection{Matrizenmultiplikation}
	\begin{equation}
		C = A B \quad C\indices{^i_k} = A\indices{^i_j} B\indices{^j_k} = \sum\limits_{j=1}^n A\indices{^i_j} B\indices{^j_k}
	\end{equation}
	\subsubsection{Spur einer Matrix}
	\begin{equation}
		\tr\nOf{A} = A\indices{^j_j} = \sum\limits_{j=1}^n A\indices{^j_j}
	\end{equation}
	\subsubsection{Transponieren einer Matrix}
	\begin{equation}
		B = A^T \quad B\indices{^i_j} = A\indices{^j_i}
	\end{equation}
	\subsection{Das Levi-Civita-Symbol}
	Wir nehmen an:
	\begin{equation}
	x = (x^1,x^2,x^3) \quad y = (y^1,y^2,y^3) \quad z = (z^1,z^2,z^3)
	\end{equation}
	Behauptung: Es wird das Kreuzprodukt $z = x \times y$ berechnet.\\
	Begründung: Komponentenweise nachrechnen:
	\begin{align}
		z^1 & = \sum\limits_{i=1}^3\sum\limits_{j=1}^3 \epsilon\indices{_i_j^1} x^i y^j \\
		& = \sum\limits_{i=1}^3 \nOf{\epsilon\indices{_i_2^1} x^iy^2 + \epsilon\indices{_i_3^1} x^i y^3} \\
		& = \epsilon\indices{_3_2^1} x^3 y^2 + \epsilon\indices{_2_3^1} x^2 y^3 \\
		& = x^2 y^3 - x^3 y^2
	\end{align}
	Analog für die anderen Komponenten (zyklische Vertauschung der Indizes)
	\begin{equation}
		z = (x^2 y^3 - x^3 y^2, x^3 y^1 - x^1 y^3, x^1 y^2 - x^2 y^1)
	\end{equation}
	\subsection{Beweisen Sie für $f, g \in \mC^1\nOf{\mR,\mR^n}$, dass}
	\begin{equation}
		\dv{}{t} \scP{f\nOf{t}}{g\nOf{t}} = \scP{\dv{}{t}f\nOf{t}}{g\nOf{t}} + \scP{f\nOf{t}}{\dv{}{t}g\nOf{t}}
	\end{equation}
	\begin{align}
		\dv{}{t} \scP{f\nOf{t}}{g\nOf{t}} & = \dv{}{t} f^i\nOf{t} g^i\nOf{t} \\
		& = \sum\limits_{i=1}^n \dv{}{t} f^i\nOf{t} g^i\nOf{t} \\
		& = \sum\limits_{i=1}^n \dv{f^i\nOf{t}}{t} g^i\nOf{t} + f^i\nOf{t} \dv{g^i\nOf{t}}{t} \\
		& = \sum\limits_{i=1}^n \dv{f^i\nOf{t}}{t} g^i\nOf{t} + \sum\limits_{i=1}^n \dv{g^i\nOf{t}}{t} f^i\nOf{t} \\
		& = \scP{\dv{}{t}f\nOf{t}}{g\nOf{t}} + \scP{f\nOf{t}}{\dv{}{t}g\nOf{t}}
	\end{align}
	\section{Einige Karten}
	\subsection{Sei $U \subset \mR^n$ eine beliebige offene Menge in der Standardtopologie. Statten Sie nun $U$ mit einer $n$-dimensionalen Karte aus.}
	\begin{equation}
		f : \mR^n \to \mR^n; x \mapsto x
	\end{equation}
	Ist eine Karte von $U$.
	\subsection{Ist diese Konstruktion auch für beliebige abgeschlossene Mengen des $\mR^n$ möglich?}
	Nein wahrscheinlich nicht. Gegenbeispiel
	\subsection{Stereographische Projektion der $S^n$ Karte. Weitere Karten für $2\nOf{n+1}$ Hemisphären $U_{i,\pm}$ für $i = 1,\dots,n+1$. Alle Hemisphären für Überdeckung? Kartenwechsel $\rightarrow$ $\mC^1$-Atlas?}
\end{document}
