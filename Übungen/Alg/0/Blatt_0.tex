\documentclass[11pt,a4paper]{scrartcl}

\newcommand{\blatt}{0}
\newcommand{\fach}{Algebra II}
\usepackage[utf8]{inputenc}
% Sonderzeichen
\usepackage{amssymb}
\usepackage{amsmath}
% Formeln, ...
\usepackage[german]{babel}
% Deutsch
\usepackage{cmbright}
% Font
\usepackage{graphicx}
% Includegraphics
\usepackage{fancyhdr}
% Pagestyle
\usepackage{titlesec}
% Titleformat
\usepackage[left=2cm,right=2cm,top=2.5cm,bottom=3cm]{geometry}
% Rand
\usepackage{xcolor}
\usepackage{color}
% Farbe in Tabellen und Plots
\usepackage{float}
% figures mit [h]
\usepackage{capt-of}
% Captions mit captionof{<figure/table>}{<Name>}
\usepackage{pgfplotstable}
% Tabellen aus .dat
\usepackage{titletoc}
% Für Inhaltsverzeichnis

\titlecontents{section}[1.5em]{}{\bfseries\contentslabel{1.5em}}{\hspace*{-2.3em}}{\titlerule*{ }\bfseries\contentspage}

\titlecontents{subsection}[2.5em]{}{}{}{\titlerule*{.}\contentspage}

\titleformat{\section}{\normalfont}{\Large\bfseries\thesection.}{8pt}{\Large\bfseries}

\titleformat{\subsection}[runin]{}{}{0pt}{\large\bfseries}
% Sections betiteln

\pagestyle{fancy}
\rhead{\begin{tabular}{l r}
  Name & Matrikelnummer\\
  Lukas Hantzko     & 10011048
\end{tabular}}
\lhead{\begin{tabular}{l}
  Physikpraktikum\\
  {\bfseries Optische Bauelemente}
\end{tabular}}
% Header

\setlength\parindent{0pt}
% Symbole
\newcommand{\mE}[1]{\text{e}^{#1}}
\newcommand{\mI}{\text{i}\,}
\newcommand{\cc}[1]{#1\!^{*}\!\,}
\newcommand{\mQ}{\mathbb{Q}}
\newcommand{\mR}{\mathbb{R}}
\newcommand{\mC}{\mathbb{C}}
\newcommand{\mN}{\mathbb{N}}
\newcommand{\mF}{\mathbb{F}}
\newcommand{\mZ}{\mathbb{Z}}
% Operatoren
\DeclareMathOperator{\gal}{Gal}
\DeclareMathOperator{\car}{char}
\DeclareMathOperator{\ord}{ord}
\DeclareMathOperator{\aut}{Aut}
\DeclareMathOperator{\fix}{Fix}
\DeclareMathOperator{\id}{id}
\DeclareMathOperator{\gl}{GL}
\DeclareMathOperator{\im}{Im}
% Klammern
\newcommand{\nOf}[1]{\left( #1 \right)}
\newcommand{\cOf}[1]{\left\{ #1 \right\}}
\newcommand{\eOf}[1]{\left[ #1 \right]}
\newcommand{\vOf}[1]{\left\lvert #1 \right\rvert}
\newcommand{\aOf}[1]{\left\langle #1 \right\rangle}


\newcommand{\sDf}[2]{\left\{ #1 \middle| #2 \right\}}
\newcommand{\mDef}{\vcentcolon =}
\newcommand{\isom}{\stackrel{\sim}{=}}
% BraKet
\newcommand{\bra}[1]{\left\langle #1 \right|}
\newcommand{\ket}[1]{\left| #1 \right\rangle}
\newcommand{\brKt}[2]{\left\langle #1 \middle| #2 \right\rangle}
\newcommand{\bkEx}[3]{\left\langle #1 \middle| #2 \middle| #3 \right\rangle}
% Operator
\newcommand{\op}[1]{\widehat{#1}}
\newcommand{\dg}[1]{\widehat{#1}\,^{\dagger}\!\,}


\begin{document}
\section{Was besagt der Hauptsatz der Galoistheorie? Geben Sie ein Anwendungsbeispiel.}
Der Hauptsatz der Galoistheorie besagt, dass die Untergruppen der Galoisgruppe einer Körpererweiterung den Zwischenkörpern der Körpererweiterung entsprechen. Ein Anwendungsbeispiel ist, Zwischenkörper zu finden.
\section{Bestimmen sie $\gal\nOf{f\nOf{x},\mQ}$ in den folgenden Fällen:}
\subsection{$f\nOf{x} = x^3-2 $}
Drei Nullstellen: $\alpha_1 = \sqrt[3]{2}$, $\alpha_2 = \sqrt[3]{2} \mE{\mI  \frac{2\pi}{3}}$, $\alpha_3 = \sqrt[3]{2} \mE{\mI 2 \frac{2\pi}{3}}$\\
Zerfällungskörper: $\mQ\nOf{\alpha_1,\alpha_2,\alpha_3}$\\
Permutation der Nullstellen (3)\\
$\rightarrow$ $\gal\nOf{f\nOf{x},\mQ} = S_3$
\subsection{$f\nOf{x} = x^4-2$}
Vier Nullstellen: $\alpha_1 = \sqrt[4]{2}$, $\alpha_2 = \sqrt[4]{2} \mE{\mI \frac{2\pi}{4}}$, $\alpha_3 = \sqrt[4]{2} \mE{\mI 2 \frac{2\pi}{4}}$, $\alpha_4 = \sqrt[4]{2} \mE{\mI 3 \frac{2\pi}{4}}$\\
Zerfällungskörper: $\mQ\nOf{\mI,\sqrt[4]{2}}$ \\
Permutation der Nullstellen (8) \\
$\rightarrow$ $\gal\nOf{f\nOf{x},\mQ} = D_4$
\subsection{$f\nOf{x} = x^4 - 4 x^2 + 2$}
Vier Nullstellen: $y=x^2$ $y_{\pm} = 2 \pm \sqrt{4-2}$ $x_{1,2,3,4} = \pm \sqrt{2 \pm \sqrt{2}}$ \\
Zerfällungskörper: $\mQ\nOf{\sqrt{2 + \sqrt{2}}}$ \\
Die Galoisgruppe ist $S_2 \times S_2$
\subsection{$f\nOf{x} = x^3 - 3x + 1$}
Nullstellen: \\
Zerfällungskörper: \\
Galoisgruppe:
\section{Für Beispiele in 2: Explizit}
\subsection{Verband der Untergruppen $\gal\nOf{f\nOf{x},\mQ}$}
\subsection{Verband der Zwischenkörper $\mQ \subset E$}
\end{document}
