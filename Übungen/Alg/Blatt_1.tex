\documentclass[11pt,a4paper]{scrartcl}

% Allgemein
\newcommand{\blatt}{1}
\newcommand{\fach}{Algebra II}
\usepackage[utf8]{inputenc}
% Sonderzeichen
\usepackage{amssymb}
\usepackage{amsmath}
% Formeln, ...
\usepackage[german]{babel}
% Deutsch
\usepackage{cmbright}
% Font
\usepackage{graphicx}
% Includegraphics
\usepackage{fancyhdr}
% Pagestyle
\usepackage{titlesec}
% Titleformat
\usepackage[left=2cm,right=2cm,top=2.5cm,bottom=3cm]{geometry}
% Rand
\usepackage{xcolor}
\usepackage{color}
% Farbe in Tabellen und Plots
\usepackage{float}
% figures mit [h]
\usepackage{capt-of}
% Captions mit captionof{<figure/table>}{<Name>}
\usepackage{pgfplotstable}
% Tabellen aus .dat
\usepackage{titletoc}
% Für Inhaltsverzeichnis

\titlecontents{section}[1.5em]{}{\bfseries\contentslabel{1.5em}}{\hspace*{-2.3em}}{\titlerule*{ }\bfseries\contentspage}

\titlecontents{subsection}[2.5em]{}{}{}{\titlerule*{.}\contentspage}

\titleformat{\section}{\normalfont}{\Large\bfseries\thesection.}{8pt}{\Large\bfseries}

\titleformat{\subsection}[runin]{}{}{0pt}{\large\bfseries}
% Sections betiteln

\pagestyle{fancy}
\rhead{\begin{tabular}{l r}
  Name & Matrikelnummer\\
  Lukas Hantzko     & 10011048
\end{tabular}}
\lhead{\begin{tabular}{l}
  Physikpraktikum\\
  {\bfseries Optische Bauelemente}
\end{tabular}}
% Header

\setlength\parindent{0pt}
% Symbole
\newcommand{\mE}[1]{\text{e}^{#1}}
\newcommand{\mI}{\text{i}\,}
\newcommand{\cc}[1]{#1\!^{*}\!\,}
\newcommand{\mQ}{\mathbb{Q}}
\newcommand{\mR}{\mathbb{R}}
\newcommand{\mC}{\mathbb{C}}
\newcommand{\mN}{\mathbb{N}}
\newcommand{\mF}{\mathbb{F}}
\newcommand{\mZ}{\mathbb{Z}}
% Operatoren
\DeclareMathOperator{\gal}{Gal}
\DeclareMathOperator{\car}{char}
\DeclareMathOperator{\ord}{ord}
\DeclareMathOperator{\aut}{Aut}
\DeclareMathOperator{\fix}{Fix}
\DeclareMathOperator{\id}{id}
\DeclareMathOperator{\gl}{GL}
\DeclareMathOperator{\im}{Im}
% Klammern
\newcommand{\nOf}[1]{\left( #1 \right)}
\newcommand{\cOf}[1]{\left\{ #1 \right\}}
\newcommand{\eOf}[1]{\left[ #1 \right]}
\newcommand{\vOf}[1]{\left\lvert #1 \right\rvert}
\newcommand{\aOf}[1]{\left\langle #1 \right\rangle}


\newcommand{\sDf}[2]{\left\{ #1 \middle| #2 \right\}}
\newcommand{\mDef}{\vcentcolon =}
\newcommand{\isom}{\stackrel{\sim}{=}}
% BraKet
\newcommand{\bra}[1]{\left\langle #1 \right|}
\newcommand{\ket}[1]{\left| #1 \right\rangle}
\newcommand{\brKt}[2]{\left\langle #1 \middle| #2 \right\rangle}
\newcommand{\bkEx}[3]{\left\langle #1 \middle| #2 \middle| #3 \right\rangle}
% Operator
\newcommand{\op}[1]{\widehat{#1}}
\newcommand{\dg}[1]{\widehat{#1}\,^{\dagger}\!\,}


% Matrizen etc.
\newcommand{\tauMat}{\left(\begin{array}{r r} 1 & 0 \\ 0 & -1\end{array}\right)}
\newcommand{\idMat}{\left(\begin{array}{r r} 1 & 0 \\ 0 & 1\end{array}\right)}
\newcommand{\sigMat}{\left(\begin{array}{r r} \cos \nOf{\phi} & \sin \nOf{\phi} \\ -\sin \nOf{\phi} & \cos \nOf{\phi} \end{array}\right)}
\newcommand{\sigiMt}{\left(\begin{array}{r r} \cos \nOf{i\cdot\phi} & \sin \nOf{i\cdot\phi} \\ -\sin \nOf{i\cdot\phi} & \cos \nOf{i\cdot\phi} \end{array}\right)}

% Text
\begin{document}
\section{$n\in \mN$, $n > 3$. $\sigma$ Drehung $\mR^2$ um $\phi = \frac{2\pi}{n}$. $\tau$ Spiegelung an $y$-Achse. $D_n = \aOf{\sigma,\tau}$}
Im folgenden wird $x\in \mR^2$ angenommen.
\subsection{Es gilt $\ord\nOf{\sigma} = n$, $\ord\nOf{\tau} = 2$ und $\sigma\tau\sigma = \tau$.}
$\ord\nOf{\sigma} = n$, da:
\begin{equation}
	\sigma^i\nOf{x} = \sigiMt \cdot x
\end{equation}
Und damit:
\begin{equation}
	\sigma^n\nOf{x} = \idMat \cdot x = x \quad \sigma^n = \id
\end{equation}
Für $\tau$:
\begin{equation}
	\tau\nOf{x} = \tauMat\cdot x \quad \tau^2\nOf{x} = \left(\begin{array}{r r} 1 & 0 \\ 0 & 1\end{array}\right)\cdot x = x \quad \tau^2 = \id
\end{equation}
Damit ist $\ord\nOf{\tau} = 2$.
\begin{equation}
	\sigma\tau\sigma\nOf{x} = \sigMat \tauMat \sigMat \cdot x = \tauMat \cdot x = \tau\nOf{x}
\end{equation}
Es gilt also auch $\sigma\tau\sigma = \tau$

\subsection{Es gilt $D_n = \sDf{\tau^i\sigma^j}{i\in\cOf{0,1}, j\in\cOf{0,1,\dots,n-1}}$.}
$D_n$ ist definiert als die kleinste Menge die $\tau$ und $\sigma$ enthält und abgeschlossen ist unter Verknüpfung. Zunächst gilt:
\begin{equation}
	M_n \mDef \sDf{\tau^i\sigma^j}{i\in\cOf{0,1}, j\in\cOf{0,1,\dots,n-1}} \subset D_n
\end{equation}
Dies ist wahr, da $\tau\in D_n\;\rightarrow\;\tau^i\in D_n\;\forall i \in \mN$, sowie analog $\sigma^j\in D_n\;\forall j \in \mN$ implizieren (jeweils über die Abgeschlossenheit unter Verknüpfung), dass $\tau^i\sigma^j\in D_n\;\forall i,j\in\mN$. Damit gilt $M_n \subset D_n$, da alle Elemente aus $M_n$ in $D_n$ enthalten sind.\\
Fehlt zu zeigen, dass $M_n$ abgeschlossen ist unter Verknüpfung, sowie dass $\sigma$ und $\tau$ in $M_n$ sind. Für letzteres reicht $j$ und $i$ als $1$ beziehungsweise $0$ zu wählen. $M_n$ ist abgeschlossen unter Verknüpfung, da:
\begin{equation}
	\tau^i\sigma^j\tau^k\sigma^l
\end{equation}
Für $k,i\in\cOf{0,1}$ und $j,l\in\cOf{0,\dots,n-1}$ Fälle:
\begin{enumerate}
	\item[$k=0$] In diesem Fall ist:
	\begin{equation}
		\tau^i\sigma^j\tau^k\sigma^l = \tau^i\sigma^{j+l} = \tau^i\sigma^{j+l \mod n} \in M_n
	\end{equation}
	\item[$k=1$] In diesem Fall ist:
	\begin{align}
		\tau^i\sigma^j\tau^k\sigma^l & = \tau^i\sigma^j\tau\sigma^l\\
		& = \tau^i\sigma^{j-1}\sigma\tau\sigma\sigma^{l-1}\\
		& = \tau^i\sigma^{j-1}\tau\sigma^{l-1}\\
		& = \dots \\
		& = \tau^{i+1}\sigma^{l-j} = \tau^{i+1 \mod 1}\sigma^{l-j \mod n} \in M_n 
	\end{align}
	Wobei jeweils $\sigma\tau\sigma = \tau$ und die Ordnungen von $\sigma$ und $\tau$ ausgenutzt worden sind.
\end{enumerate}

\subsection{$D_n$ hat $2n$ Elemente und ist nicht kommutativ.}
$D_n$ ist nicht kommutativ, da $\tau\sigma\tau = \id \neq \tau\tau\sigma$. $D_n$ hat $2n$ Elemente, aufgrund der in $M_n$ gewählten Darstellung.

\section{$K$ Körper und $f\nOf{x}\in K \eOf{x}$ irreduzibel, normiert. $a$ Nullstelle von $f\nOf{x}$ in Erweiterungskörper von $K$.}

\subsection{Beweisen Sie: Ist auch $f\nOf{a + 1} = 0$, so gilt $\car\nOf{K} > 0$.}
Es gilt:
\\ Gelte nun weiter $\car\nOf{K} = p$ und $a^p - a \in K$.
\subsection{Beweisen Sie, dass $f\nOf{x} = x^p - x - \nOf{a^p - a}$ gilt.}

\subsection{Beweisen Sie, dass die Erweiterung $K\nOf{a}/K$ galoissch ist.}

\subsection{Beweisen Sie, dass $\aut\nOf{K\nOf{a};K}$ zyklisch von Ordnung $p$ ist.}

\section{$\mC\nOf{x}$ rationale Funktionen über $\mC$. In $\aut\nOf{\mC\nOf{x};\mC}$, betrachte $\sigma$, $\tau$ mit $\sigma\nOf{x} = - x$ und $\tau\nOf{x} = \mI x^{-1}$. $G = \aOf{\sigma,\tau}\subset \aut\nOf{\mC\nOf{x};\mC}$.}

\subsection{Beweisen Sie, dass $G$ endlich ist. Welche Ihnen bekannte Gruppe ist $G$?}
Zwei selbstinverse Elemente als Erzeuger -> $S^2 \times S^2$

\subsection{Beweisen Sie, dass $\fix\nOf{\mC\nOf{x};G}$ rationaler Fkt-Körper über $\mC$; $\fix\nOf{\mC\nOf{x};G} = \mC\nOf{y}$ mit $y\in\mC\nOf{x}$. $y$ angeben.}	

\end{document}
