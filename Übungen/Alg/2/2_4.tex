\section{Sei $n\geq 2 \in\;\mN$, $a\in\mQ$ und $E$ der Zerfällungskörper von $f\nOf{x}=x^n-a$ über $\mQ$.}
\subsection{Betrachten Sie die folgende multiplikative Gruppe von Matrizen $G_n$, beweisen Sie: Es gibt einen injektiven Gruppenhomomorphismus}
\begin{equation}
	\Psi:\aut\nOf{E;\mQ} \hookrightarrow G_n\mDef\sDf{\nOf{\begin{array}{cc}r&s\\0&1\end{array}}}{r\in\mZ^*_n,s\in\mZ_n}\subset\gl\nOf{2,\mZ_n}
\end{equation}
Die Nullstellen werden durch Elemente der Galoisgruppe permutiert. Dabei gibt es einmal die Rotationen, die die Nullstellen einen weiter drehen

\subsection{Bestimmen Sie $\im\nOf{\Psi}$ für $n=10$ und $a=5$, das heißt für das Polynom $x^{10}-5\in\mQ\eOf{x}$.}
Da der Zerfällungskörper von Ordnung $\phi\nOf{n}\cdot n$ ist, ist die Ordnung der beiden Gruppen gleich. Damit ist das Bild die gesamte $G_10$.