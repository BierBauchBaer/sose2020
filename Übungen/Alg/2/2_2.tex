\section{Sei $L = \mQ\nOf{\sqrt{2},\sqrt{3}}$.}
\subsection{Zeigen Sie, dass $L$ galoissch über $\mQ$ ist.}
$L$ ist der Zerfällungskörper des Polynoms
\begin{equation}
	f\nOf{x} = \nOf{x^2-2}\nOf{x^2-3} \in \mQ\eOf{x}
\end{equation}
welches separabel ist, da alle Nullstellen einfach sind ($\pm\sqrt{2}$, $\pm\sqrt{3}$) und folglich ist die Körpererweiterung galoissch.


\subsection{Bestimmen Sie die Galoisgruppe $G\nOf{L/\mQ}$.}
Es gibt vier Elemente der Galoisgruppe:
\begin{align}
	\varphi_1\nOf{\sqrt2} = -\sqrt2 & \quad \varphi_1\nOf{\sqrt3} = \sqrt3 \\
	\varphi_2\nOf{\sqrt2} = \sqrt2 & \quad \varphi_2\nOf{\sqrt3} = -\sqrt3 \\
	\varphi_3\nOf{\sqrt2} = -\sqrt2 & \quad \varphi_3\nOf{\sqrt3} = -\sqrt3 \\
	\id\nOf{\sqrt2} = \sqrt2 & \quad \id\nOf{\sqrt3} = \sqrt3 
\end{align}
Andere Elemente gibt es nicht, da die Galoisgruppe die Nullstellen des Polynoms permutiert. Abbildungen $\phi$ die zum Beispiel $\sqrt2$ auf $\sqrt3$ abbilden sind keine Automorphismen auf $\mQ$, da
\begin{equation}
	3 = \phi\nOf{3} = \phi\nOf{\sqrt3\sqrt3} = \phi\nOf{\sqrt3}\phi\nOf{\sqrt3} = \sqrt2\sqrt2 = 2
\end{equation}
Die Galoisgruppe ist isomorph zur kleinschen Vierergruppe.


\subsection{Bestimmen Sie alle Zwischenkörper.}
Die Zwischenkörper sind $\mQ\nOf{\sqrt{3}}$, $\mQ\nOf{\sqrt{2}}$ und $\mQ\nOf{\sqrt{6}}$, die Fixkörper der Untergruppen erzeugt von $\varphi_1$, $\varphi_2$ und $\varphi_3$. $\sqrt6$ ist Fix, da
\begin{equation}
	\sqrt6 = \sqrt3\sqrt2 = \nOf{-\sqrt3}\nOf{-\sqrt2} = \varphi_3\nOf{\sqrt3\sqrt2} = \varphi_3\nOf{\sqrt6}
\end{equation}
Es gibt keine anderen Untergruppen (außer $\cOf{\id}$ und $G\nOf{L/\mQ}$)